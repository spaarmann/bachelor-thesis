\documentclass{article}

% Times font, math in Times Italic
\usepackage{mathptmx}
\usepackage{mathtools}
\usepackage{amsfonts}
% Generate PDF links for ToC, cite, ref, etc.
\usepackage{hyperref}

% This somehow takes advantasge of pdftex features and magically make the document typeset nicer,
% for example it's managed to resolve an underfull hbox warning on its own.
\usepackage{microtype}

% Prevent single first or last lines of a paragraph appearing alone at the top or bottom
% of a page.
\clubpenalty=10000
\widowpenalty=10000

% We get a bunch of underfull hboxes is the references, which seem hard to impossible to fix nicely,
% so this silences them for now so they're not spamming the output.
% TODO: Before actual submission, turn this off and try and fix any underfull hbox reported. If some
% can't be fixed, that's not big deal either.
\hbadness 10000

% This declares \abs and \norm commands
\DeclarePairedDelimiter\abs{\lvert}{\rvert}%
\DeclarePairedDelimiter\norm{\lVert}{\rVert}%

% Swap the definition of \abs* and \norm*, so that \abs
% and \norm resizes the size of the brackets, and the 
% starred version does not.
\makeatletter
\let\oldabs\abs
\def\abs{\@ifstar{\oldabs}{\oldabs*}}
%
\let\oldnorm\norm
\def\norm{\@ifstar{\oldnorm}{\oldnorm*}}
\makeatother

% Number definitions, theorems, and lemmas with a single counter.
\newtheorem{theorem}{Theorem}[section]
\newtheorem{lemma}[theorem]{Lemma}
\newtheorem{definition}[theorem]{Definition}

\title{Thesis}
\author{Sebastian Paarmann}
\date\today

\begin{document}

\maketitle

\tableofcontents

\section{Introduction}

TODO

% TODO: The separation between "Introduction", "The Cluster Editing Problem", and "Previous Work"
% needs some more thought; may not even want to have all 3.

\section{The Cluster Editing Problem}

A \emph{cluster} graph, also called \emph{transitive} graph, is a graph which consists only of
disjoint cliques. The \textsc{Cluster Editing} problem asks to transform an input graph into such a
cluster graph, using a minimal number of edge \emph{edits} (insertions and deletions). For the
\textsc{Weighted Cluster Editing} problem, the input is a weighted graph (with weights for both
edges and non-edges) with the weights representing the cost of inserting a non-edge or deleting an
edge. The optimal solution then has minimal cost instead of minimal count of edits. We will start by
considering the unweighted problem, but eventually deal mostly with the weighted version.

% TODO: This introduction also needs some more work, could be a bit more clear and we also need a
% formal formulation of the problem somewhere (symmetric difference of edge sets etc.)
% TODO: Many papers also directly formulate the problem as having $k$ as input, maybe mention/do
% something with that? Use it ourself?

% TODO: Motivation for the problem, practical applications

First, some notation used throughout: A graph $G = (V, E)$ has a vertex set $V$ and edge set $E
\subset \binom{V}{2}$. Graphs are simple graphs, i.e.\ they contain no self-loops or parallel edges.
We abbreviate an edge or non-edge $\{u, v\} \in \binom{V}{2}$ as $uv$.  $N_G(v)$ is the (open)
neighborhood of a vertex $v$ in $G$. Similarly, $N_G[v]$ is the closed neighborhood of~$v$, i.e.\
$N_G(v) \cup \{v\}$. If it is clear which graph is meant, we may also just write $N(v)$ and $N[v]$.
A weighted graph $G$ is characterized by a weight function $s\colon \binom{V}{2} \to \mathbb{Z}$. We
consider $uv$ to be an edge if $s(uv) > 0$ and $uv$ to be a non-edge if $s(uv) \leq 0$.
% TODO: Make sure all relevant notation is included here, and we only have notation here that we end
% up actually using.

\textsc{Cluster Editing} is NP-hard/complete(?).
% TODO: Cite, APX-hardness, existing approximations(?)

% TODO: fpt approaches and fpt-tractability, including how you keep increasing k etc.

A very simple FPT algorithm is given by the observation that a graph $G = (V, E)$ is a cluster graph
iff.\ there are no 3 vertices $v, u, w \in V$ with $uv, uw \in E, vw \notin E$. Such a set of
vertices is called a \emph{conflict triple}. The problem can be solved by recursively searching for
a conflict triple and branching into three cases: Add $vw$, remove $uv$, or remove $uw$. These are
the only ways of resolving a conflict, and once there are no more conflicts, a solution can be
returned. With an upper bound $k$ for the number of edits as described above, this gives a search
tree of size $3^k$.

% TODO: Image for a conflict, and its resolutions, I think

% TODO: Should decide on consistent rules for using O(k) vs O(k + n + m) etc., and apply these to
% existing text.

\section{Previous Work}

% TODO: There are still a few things missing from this:
% - Improvements over time to problem kernels
% - ILP formulations
% - Whatever introduced that weird impractical 1.92^k algorithm
% - Work after the golden ratio paper, including e.g. the paper by Hartung and Hoos, among others.

Clustering problems in general have a very long history, so we will focus only on the specific
\textsc{Cluster Editing} problem as formulated above. In 1999, Ben-Dor, Shamir, and Yakhini
introduced the unweighted problem in order to perform clustering on gene expression patterns
\cite{BenDor}. They provided a stochastic model for the corruption of the true clustering introduced
by measurements and present an $O(n^2 log(n)^c)$ algorithm that reconstructs the true clustering
with high probability, as well as a heuristic approach based on a similar idea. Many other
approaches for clustering gene expression data had been proposed and investigated already at this
time, see \cite{ShamirOverview} for an overview from 2001.

In \cite{ShamirModifications} Shamir, Sharan, and Tsur investigated \textsc{Cluster Editing} as well
as the related \textsc{Cluster Completion} (only edge insertions allowed) and \textsc{Cluster
Deletion} (only edge deletions allowed) problems further. They showed that \textsc{Cluster
Completion} can be solved in polynomial time, \textsc{Cluster Deletion} is NP-hard to approximate to
within some constant factor, and, most relevant for us, \textsc{Cluster Editing} is NP-complete.
Additionally they also considered variants of the problems with a fixed number of clusters as input,
which we will not further consider. In \cite{Gramm}, Gramm et al.\ mention that the NP-completeness
of \textsc{Cluster Editing} can already be derived from \cite{Krivanek}, published in 1986.

Bansal, Blum, and Chawla independently researched a group of problems they call \textsc{Correlation
Clustering} in \cite{Bansal}. In their formulation, "minimizing disagreements" between the input
labelling and the output clustering is equivalent to the unweighted \textsc{Cluster Editing}
problem, which they give a constant-factor approximation for. This result was improved on in 2005 by
Charikar, Guruswami, and Wirth in \cite{Charikar}, who show the problem to be APX-hard and give an
approximation with a constant factor of 4.

In 2003 Gramm et al.\ gave initial results regarding fixed-parameter approaches for \textsc{Cluster
Editing} specifically in \cite{Gramm}. Before that, in \cite{Cai}, Cai investigated fixed-parameter
tractability of graph modification problems characterized by forbidden subgraphs in a more general
sense, the results of which imply an $O(k^3 * \abs{G}^4)$ algorithm for \textsc{Cluster Editing}.
Gramm et al.\ gave an $O(k^3)$ kernel for the problem, as well as a straightforward $O(k^3 +
\abs{V}^3)$ fixed-parameter algorithm. Additionally they show a more advanced branching strategy
that improves the search tree size to $O(2.27^k)$, leading to an $O(2.27^k + \abs{V}^3)$ algorithm.
This approach was implemented in practice for the first time and empirically compared to an LP-based
solver by Dehne et al.\ in 2006 \cite{Dehne}. They find that the $O(2.27^k)$ is indeed faster in
practice than the simpler $O(k^3)$ branching.

Another avenue of research was started in 2007 by Rahmann et al.\ with \cite{Rahmann}. This
introduced the first fixed-parameter algorithm for the weighted version of the problem, along with
several data reduction rules to speed it up. They also introduced two heuristic algorithms.

Building on this, Böcker et al.\ published several papers in 2008: \cite{AnApproach},
\cite{GoingWeighted}, and \cite{ExactAlgos}. In \cite{AnApproach} they introduce a variant of the
refined branching strategy from \cite{Gramm} for the weighted problem, resulting in an $O(2.42^k)$
algorithm. Most importantly, this is the first paper to introduce merging two vertices, which is an
operation only possible on the weighted problem. As a result, with merging, Böcker et al.\ find the
simpler $O(3^k)$ branching faster in practice than the refined branching strategy, largely due to
the significant effect of merging. Merging is also the basis for a very simple $O(2^k)$ branching
algorithm introduced in \cite{GoingWeighted} which means the fastest practical approach to solving the
unweighted problem is now to transform it into a weighted instance and solve that.
\cite{GoingWeighted} also immediately improves upon this with an $O(1.82^k)$ algorithm that is
achieved by more carefully choosing which edge to branch on. This beats even the previous best
theoretical runtime for the unweighted problem of $O(1.92^k)$. Lastly, in \cite{ExactAlgos}, Böcker
et al.\ perform practical experiments with these algorithms, comparing them with a cut-and-branch
based ILP implementation. They find both approaches to be competitive, with different performance
characteristics. Notably, good data reduction rules make the FPT approach feasible even for graphs
that require several thousand edge modifications, despite the worst-case running times.

This basic approach was improved further in \cite[Böcker and Damaschke, 2011]{EvenFaster} and
\cite[Böcker, 2012]{GoldenRatio}. First, a theorem introduced by Damaschke in \cite{BoundedDegree}
concerning the structure of graphs in which no edge is part of three conflict triples, is used to
improve the algorithm to $O(1.76^k)$. Then zero-edges (which will be discussed in more detail later)
can be used to show additional structural properties, allowing even more efficient branching for an
$O(1.62^k)$ search tree.

\begin{thebibliography}{99}

% TODO: It would probably be good to double-check all of these are in fact correct ^^'
% TODO: And double-check the format too. Certainly a bunch of the abbreviation dots produce too-long
% spaces :/
% For now, these are just taken straight out of Zotero

\bibitem{BenDor}
A. Ben-Dor, R. Shamir, and Z. Yakhini, “Clustering Gene Expression Patterns,” Journal of
Computational Biology, vol. 6, no. 3–4, pp. 281–297, Oct. 1999, doi: 10.1089/106652799318274.

\bibitem{ShamirOverview}
R. Shamir and R. Sharan, “Algorithmic Approaches to Clustering Gene Expression Data,” in Current
Topics in Computational Biology, 2001, pp. 269–300.

\bibitem{ShamirModifications}
R. Shamir, R. Sharan, and D. Tsur, “Cluster graph modification problems,” Discrete Applied
Mathematics, vol. 144, no. 1, pp. 173–182, Nov. 2004, doi: 10.1016/j.dam.2004.01.007.

\bibitem{Krivanek}
M. Křivánek and J. Morávek, “NP-hard problems in hierarchical-tree clustering,” Acta Informatica,
vol. 23, no. 3, pp. 311–323, Jun. 1986, doi: 10.1007/BF00289116.

\bibitem{Gramm}
J. Gramm, J. Guo, F. Hüffner, and R. Niedermeier, “Graph-Modeled Data Clustering: Fixed-Parameter
Algorithms for Clique Generation,” in Algorithms and Complexity, vol. 2653, R. Petreschi, G.
Persiano, and R. Silvestri, Eds. Berlin, Heidelberg: Springer Berlin Heidelberg, 2003, pp. 108–119.

\bibitem{Bansal}
N. Bansal, A. Blum, and S. Chawla, “Correlation Clustering,” Machine Learning, vol. 56, no. 1,
pp. 89–113, Jul. 2004, doi: 10.1023/B:MACH.0000033116.57574.95.

\bibitem{Charikar}
M. Charikar, V. Guruswami, and A. Wirth, “Clustering with qualitative information,” Journal of
Computer and System Sciences, vol. 71, no. 3, pp. 360–383, Oct. 2005, doi:
10.1016/j.jcss.2004.10.012.

\bibitem{Cai}
L. Cai, “Fixed-parameter tractability of graph modification problems for hereditary properties,”
Information Processing Letters, vol. 58, no. 4, pp. 171–176, May 1996, doi:
10.1016/0020-0190(96)00050-6.

\bibitem{Dehne}
F. Dehne, M. A. Langston, X. Luo, S. Pitre, P. Shaw, and Y. Zhang, “The Cluster Editing Problem:
Implementations and Experiments,” in Parameterized and Exact Computation, Berlin, Heidelberg, 2006,
pp. 13–24, doi: 10.1007/11847250\_2.

\bibitem{Rahmann}
S. Rahmann, T. Wittkop, J. Baumbach, M. Martin, A. Truß, and S. Böcker, “Exact and Heuristic
Algorithms for Weighted Cluster Editing,” in Computational Systems Bioinformatics, University of
California, San Diego, USA, Sep. 2007, pp. 391–401, doi: 10.1142/9781860948732\_0040.

\bibitem{AnApproach}
S. Böcker, S. Briesemeister, Q. A. Bui, and A. Truß, “A Fixed-Parameter Approach for Weighted
Cluster Editing,” 2008, doi: 10.1142/9781848161092\_0023.

\bibitem{GoingWeighted}
S. Böcker, S. Briesemeister, Q. B. A. Bui, and A. Truss, “Going Weighted: Parameterized
Algorithms for Cluster Editing,” in Combinatorial Optimization and Applications, Berlin, Heidelberg,
2008, pp. 1–12, doi: 10.1007/978-3-540-85097-7\_1.

% TODO: We mention this as being from 2008 in the text, should cite the actual version published
% then probably.
\bibitem{ExactAlgos}
S. Böcker, S. Briesemeister, and G. W. Klau, “Exact Algorithms for Cluster Editing: Evaluation
and Experiments,” Algorithmica, vol. 60, no. 2, pp. 316–334, Jun. 2011, doi:
10.1007/s00453-009-9339-7.

\bibitem{EvenFaster}
S. Böcker and P. Damaschke, “Even faster parameterized cluster deletion and cluster editing,”
Information Processing Letters, vol. 111, no. 14, pp. 717–721, Jul. 2011, doi:
10.1016/j.ipl.2011.05.003.

\bibitem{GoldenRatio}
S. Böcker, “A golden ratio parameterized algorithm for Cluster Editing,” Journal of Discrete
Algorithms, vol. 16, pp. 79–89, Oct. 2012, doi: 10.1016/j.jda.2012.04.005.

\bibitem{BoundedDegree}
P. Damaschke, “Bounded-Degree Techniques Accelerate Some Parameterized Graph Algorithms,” in
Parameterized and Exact Computation, vol. 5917, J. Chen and F. V. Fomin, Eds. Berlin, Heidelberg:
Springer Berlin Heidelberg, 2009, pp. 98–109.


\end{thebibliography}

\end{document}
