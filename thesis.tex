\documentclass{article}

% Times font, math in Times Italic
\usepackage{mathptmx}
\usepackage{amsfonts}
% Generate PDF links for ToC, cite, ref, etc.
\usepackage{hyperref}

% Prevent single first or last lines of a paragraph appearing alone at the top or bottom
% of a page.
\clubpenalty=10000
\widowpenalty=10000

% Number definitions, theorems, and lemmas with a single counter.
\newtheorem{theorem}{Theorem}[section]
\newtheorem{lemma}[theorem]{Lemma}
\newtheorem{definition}[theorem]{Definition}

\title{Thesis}
\author{Sebastian Paarmann}
\date\today

\begin{document}

\maketitle

\tableofcontents

\section{The Cluster Editing Problem}

A \emph{cluster} graph, also called \emph{transitive} graph, is a graph which consists
only of disjoint cliques. The \textsc{Cluster Editing} problem asks to transform an input
graph into such a cluster graph, using a minimal number of edge \emph{edits} (insertions
and deletions). For the \textsc{Weighted Cluster Editing} problem, the input is a weighted
graph (with weights for both edges and non-edges) with the weights representing the cost
of inserting a non-edge or deleting an edge. The solution should then require minimal cost
instead of minimal count of edits. Oops, a misspelt word!

\end{document}
